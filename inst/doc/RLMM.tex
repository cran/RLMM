% -*- mode: noweb; noweb-default-code-mode: R-mode; -*-
%\VignetteIndexEntry{RLMM doc}
%\VignetteKeywords{Preprocessing, Affymetrix}
%\VignetteDepends{MASS, RLMM}
%\VignettePackage{RLMM}
%documentclass[12pt, a4paper]{article}
\documentclass[12pt]{article}

\usepackage{amsmath,pstricks}
\usepackage{hyperref}
\usepackage[authoryear,round]{natbib}

\textwidth=6.2in
\textheight=8.5in
%\parskip=.3cm
\oddsidemargin=.1in
\evensidemargin=.1in
\headheight=-.3in

\newcommand{\scscst}{\scriptscriptstyle}
\newcommand{\scst}{\scriptstyle}
\newcommand{\Rfunction}[1]{{\texttt{#1}}}
\newcommand{\Robject}[1]{{\texttt{#1}}}
\newcommand{\Rpackage}[1]{{\textit{#1}}}

\author{Nusrat Rabbee and Gary Wong}
\usepackage{/usr/lib/R/share/texmf/Sweave}
\begin{document}
\title{RLMM - Robust Linear Model with Mahalanobis Distance Classifier}
\date{September 2, 2005}

\maketitle
\tableofcontents

\section{Introduction}
\noindent  
The RLMM package is a genotype calling algorithm for Affymetrix SNP arrays with a classification algorithm based on multi-chip,
multi-SNP approach for Affymetrix SNP arrays. This package uses a supervised method of classifying Affymetrix SNP array data, 
and works by using a large training sample where genotype labels are known to obtain a more accurate classification on new data.
The training set contains publicly available genotypes available for 90 Centre du Etude Polymorphisme Human(CEPH) individuals from 
the HapMap project. 

\medskip
This package contains three main functions for classification and one for plotting. 

\begin{enumerate}
\item We start out with normalization of probe data in ascii format ({\it .raw files}) with function {\bf normalize\_Rawfiles}, 
in order to get corresponding norm files ({\it .norm files}). These norm files are used to scale the new data to the scale of the training set.
Thus the normalized probe intensities are stored in the norm files. If the user has CEL files instead, there is a section below on how to
convert these files to .raw files. 

\item The next step is allele summarization which involves calling function {\bf create\_Thetafile}, which is calculating estimates of
theta A and theta B for each chip, for each SNP, based on the normalized intensities stored in the .norm files. The theta A and
theta B values are summary measures of probe intensities for each allele. Only perfect match, PMA and PMB, values are used for
the theta values.

\item The last step is Classification using Mahalanobis distances. This involves calling the {\bf Classify} function and using a
{\it regions file} (we provide this as an internal file) obtained from the training data and a {\it theta file} obtained from step 2.
\end{enumerate}

\section{Instructions for Genotyping Affymetrix Mapping 100K array - Xba set}

We will cover everything needed to classify probe data starting with installation and internal files that the algorithm depends on.
Before even starting though, make sure that you are using {\it Raw files} (ascii versions of {\it CEL files}). Also at this very 
moment, this algorithm works only for 100K - Xba data set. It will work with Hind data once we obtain the data to make a training set,
and corresponding internal files.

\subsection{Installation}
Once the package is downloaded, install the package on a UNIX machine using the following commands in the console:

\medskip
\noindent
{\it R CMD INSTALL -l /dir/mylocation RLMM.tar.gz}
\medskip

\noindent
After the -l, please put in the pathname where the package is to be stored. When installing, the part between R CMD INSTALL and
RLMM.tar.gz may be disregarded if you have the permissions to write to the main library in R-Home. For more information
see the R-admin pdf from CRAN on installing libraries. Currently, this package is working under Unix.
(We will shortly release a version that will work under Windows) Once this package is installed onto the Unix/Linux system, 
it can be accessed by R console by typing:

\medskip
R$>$ library(RLMM)
\medskip

\noindent
libDir is the pathname for the director that the package is stored. Also the functions require {\it internal files} to work properly. 
We advise that you save these files in the same directory as your working directory, so that it can be referred to by the package,
when doing classification. Presently, these files are stored under the /inst directory in this zipped package.
List of internal files: {\it Xba.CQV} and the {\it Xba.regions} file.

\subsection{Converting CEL files to .raw files (ascii)}
Affymetrix has a tool available which converts your {\it .CEL files to .raw files}. This tool is available for download from our 
website {\footnote {http://www.stat.berkeley.edu/users/nrabbee/RLMM}}. After downloading this Affymetrix tool 
onto your UNIX/Linux system, type at the unix system prompt:

\medskip
\noindent
\textdollar gtype\_cel\_to\_pq -cdf Mapping 50K\_Xba240.CDF NA06985\_Xba\_B5\_4000090.CEL
\medskip

\noindent
If this command works, you will get { \it NA06985\_Xba\_B5\_4000090.raw} file created in your directory. 
Do this for all the {\it .CEL files} in your directory. You can download this tool and the {\it .CDF files}
from our website {\footnote {http://www.stat.berkeley.edu/users/nrabbee/RLMM}}.

\subsection{Quick Start}
This section will go through the usage of basic functions that are packaged with the library RLMM (pronounced realm) 
to get you up and running as fast as possible, from probe level data (intensity values) to classification of SNPs.

\medskip
\noindent
Before starting please do the following:

\begin{itemize}
\item Make sure that the {\it .raw files} are all in one directory ({\it probefiledir}) and both 
internal files ({\it Xba.CQV and Xba.regions}) are together in a directory (same or different than { \it probefiledir}). 
We suggest that you keep the files together in a directory and set directory to that location at the beginning of the R session.
\item Load the library {\bf RLMM} by going to R console and type:
\end{itemize}

\medskip
R> library(RLMM)

\medskip
\noindent
{\bf Function Calls}

\medskip
\noindent
{\bf Reading and Normalizing the Probe Data}
\smallskip

\noindent
Probe intensity data can be read in and normalized in one step by using the {\bf normalize\_Rawfiles} function. This function will 
normalize (translate to the same scale as the training data), each corresponding {\it .raw file} and give a {\it .norm} file (ascii). 
Do not be alarmed if it takes some time, the time it takes to finish depends on the number of {\it raw} files. 

\medskip
\noindent
In R:

\smallskip
\noindent
$>$ normalize\_Rawfiles(cqvfile="", probefiledir = getwd())
\medskip

\noindent
NOTE: State the name of the {\it cqvfile} (e.g., {\it Xba.CQV}) and location of {\it probe data}, i.e.,.{\it raw} files
in probefiledir. The first parameter is required. The second parameter needs to be specified only if the {\it .raw files}
are not in the working directory. 

\medskip
\noindent
{\bf Getting theta estimates (estimated probe intensity for allele A and B)}  
\smallskip

\noindent
To get theta estimates for theta A and theta B for each SNP and chip, we need to use the {\bf create\_Thetafile} function. In the end,
the estimates will be stored in a ascii text format of a name of your own choosing ({\it thetafile}). To invoke this function, run
the following command below in R:

\medskip
\noindent
$>$ create\_Thetafile(probefiledir = getwd(),start=1,end=-1,thetafile="")

\medskip
\noindent
The parameter {\it thetafile} is required to be filled in. This will create the theta file where start specifies the 1st SNP and end 
specifies the last SNP, if the user wants to specify a subset of SNPs to be worked on (e.g., start=1 end=1000).
By default, end = -1, which makes the function process all SNPs in the .norm files (e.g., 58,960 SNPs in the Xba set).

\medskip
\noindent
{\bf Classification}
\smallskip

\noindent
Classification is done by a Mahalanobis distance classifier after genotype group centers and variance-covariance matrices are 
determined by training data. For this important step, you must call function {\bf Classify}. Here, you will need our internal
file, e.g., {\it Xba.regions}. {\it Call rate} allows the change of the cut-off value to make more accurate calls. 
Currently, eligible call rates are: 80,82,84,86,88,90,92,94,96,98,100. If you don't specify it, the default is 100%. 
So, all calls are made. If you specify a ineligible call rate (e.g., 91), the call rate will be set to 80% (which is the minimum).
Note, with lower call rate, higher accuracy is achieved.

\medskip
\noindent
$>$ Classify(genotypefile="",regionsfile="",thetafile="",callrate=100)

\medskip
\noindent
{\bf Allele Summary Plots}
\smallskip

\noindent

This is useful for exploratory purposes and to see visually how tight each cluster of genotypes, a particular SNP exhibits.
To create the theta plots we need to specify a genotype file and a theta file. 
In the example below, we set {\it Pick.Obj} equal to {\it FALSE} (NOTE: It should always be set to false at this point with RLMM ver 0.7).
{\it Snpsfilename} is a vector of snps we wish to plot, ideally each SNP is listed as a newline as a text file. Running the command 
below will save the plot in plots.ps, but if {\it plotfilename} is left blank "", it will display the graph onscreen.

\medskip
\noindent
$>$ plot\_theta(genotypefile="",thetafile="",Pick.Obj="FALSE",plotfile="plots.ps",snpsfile="snps.lst")

\smallskip
\section{Where can I learn more about RLMM?}
\noindent
Updated information on RLMM will be available at our website (see references). This is the main site where all information 
pertaining to RLMM will be stored including updates and new files for the package if necessary. For other packages and 
information relating to bioinformatics, check out Bioconductor {\footnote {http://www.bioconductor.org/}} project. 


\medskip
\noindent
Contact information for Nusrat Rabbee is $<$nrabbee@post.harvard.edu$>$ and Gary Wong is $<$wongg62@gmail.com$>$

\appendix
\section{Previous Release Notes}

\section{Future Plans}

\begin{itemize}
\item Add updates to package to handle 10K data set
\item Add examples and more documentation
\item Make the package more user friendly
\end{itemize}

\smallskip
\noindent
\section {References}
\noindent
The pre-print of the manuscript containing a description of RLMM is available  at the RLMM website http://www.stat.berkeley.edu/users/nrabbee/RLMM. 

\end{document}
